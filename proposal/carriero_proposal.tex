% Options for packages loaded elsewhere
\PassOptionsToPackage{unicode}{hyperref}
\PassOptionsToPackage{hyphens}{url}
%
\documentclass[
  11pt,
]{article}
\usepackage{amsmath,amssymb}
\usepackage{lmodern}
\usepackage{iftex}
\ifPDFTeX
  \usepackage[T1]{fontenc}
  \usepackage[utf8]{inputenc}
  \usepackage{textcomp} % provide euro and other symbols
\else % if luatex or xetex
  \usepackage{unicode-math}
  \defaultfontfeatures{Scale=MatchLowercase}
  \defaultfontfeatures[\rmfamily]{Ligatures=TeX,Scale=1}
\fi
% Use upquote if available, for straight quotes in verbatim environments
\IfFileExists{upquote.sty}{\usepackage{upquote}}{}
\IfFileExists{microtype.sty}{% use microtype if available
  \usepackage[]{microtype}
  \UseMicrotypeSet[protrusion]{basicmath} % disable protrusion for tt fonts
}{}
\makeatletter
\@ifundefined{KOMAClassName}{% if non-KOMA class
  \IfFileExists{parskip.sty}{%
    \usepackage{parskip}
  }{% else
    \setlength{\parindent}{0pt}
    \setlength{\parskip}{6pt plus 2pt minus 1pt}}
}{% if KOMA class
  \KOMAoptions{parskip=half}}
\makeatother
\usepackage{xcolor}
\usepackage[margin=25mm]{geometry}
\usepackage{graphicx}
\makeatletter
\def\maxwidth{\ifdim\Gin@nat@width>\linewidth\linewidth\else\Gin@nat@width\fi}
\def\maxheight{\ifdim\Gin@nat@height>\textheight\textheight\else\Gin@nat@height\fi}
\makeatother
% Scale images if necessary, so that they will not overflow the page
% margins by default, and it is still possible to overwrite the defaults
% using explicit options in \includegraphics[width, height, ...]{}
\setkeys{Gin}{width=\maxwidth,height=\maxheight,keepaspectratio}
% Set default figure placement to htbp
\makeatletter
\def\fps@figure{htbp}
\makeatother
\setlength{\emergencystretch}{3em} % prevent overfull lines
\providecommand{\tightlist}{%
  \setlength{\itemsep}{0pt}\setlength{\parskip}{0pt}}
\setcounter{secnumdepth}{-\maxdimen} % remove section numbering
\ifLuaTeX
  \usepackage{selnolig}  % disable illegal ligatures
\fi
\usepackage[]{biblatex}
\addbibresource{carriero\_proposal.bib}
\nocite{bow, no_ml, stat_v_ml, myths, mcc, lp, hotels, kaur, stop, to_smote, pre_ensemble, rusboost, reality, old_wine}
\IfFileExists{bookmark.sty}{\usepackage{bookmark}}{\usepackage{hyperref}}
\IfFileExists{xurl.sty}{\usepackage{xurl}}{} % add URL line breaks if available
\urlstyle{same} % disable monospaced font for URLs
\hypersetup{
  pdftitle={Proposal: \textasciitilde{} insert title \textasciitilde{} .},
  pdfauthor={Alex Carriero (9028757) },
  hidelinks,
  pdfcreator={LaTeX via pandoc}}

\title{\textbf{Proposal: \textasciitilde{} insert title
\textasciitilde{} .}}
\author{\textbf{Alex Carriero (9028757) \vspace{4in}}}
\date{}

\begin{document}
\maketitle

Date: 13/10/2022\\
\strut \\
Word Count: XXX\\
\strut \\
Program: Methodology and Statistics for Behvarioual, Biomedical, and
Social Sciences.\\
\strut \\
Supervisors: Maarten van Smeden (Utrecht Medical Center Utrecht), Ben
van Calster (Leuven University and Leiden University Medical Center) and
Kim Luijken (Utrecht Medical Center Utrecht).

Host Institution: Julius Center for Health Science and Primary Care,
UMC.\\
\strut \\
Candidate Journal: Statistics in Medicine.\\
\strut \\
FETC-approval: 22-1809 (pending)

\newpage

\hypertarget{introduction}{%
\subsubsection{1 \textbar{} Introduction}\label{introduction}}

~~~~~The presence of prediction modelling in the field of medicine is
rapidly increasing. In a clinical setting, the goal of prediction
modelling is often to (accurately) predict a patient's risk of
experiencing an event (e.g., a stroke). In other words, to predict which
binary class (event or no event) a patient is most likely to belong to.
Due to the (thankfully) rare nature of many diseases, the data available
to train clinical prediction models are often heavily imbalanced (i.e.,
the number of patients in one class dramatically outnumbers the other)
\autocite{summary_m}. This is referred to as class imbalance, and is
seen as a major problem in the field of machine learning
\autocite{summary_m}.

~~~~~Class imbalance is thought to diminish the quality of prediction
models \autocite{yu}. Quality is multifaceted. It is characterized by
three criteria: accuracy, discrimination, and calibration. Accuracy
refers to the proportion of patients that a model classifies correctly
(after a risk threshold is imposed). Discrimination refers to a model's
ability to yield higher risk estimates for patients in the positive
class than for those in the negative class. Finally, calibration refers
to the reliability of the risk predictions themselves; for instance, a
poorly calibrated model may produce risk predictions that consistently
over- or under-estimate reality, or produce risk estimates which are too
extreme (too close of 0 or 1) or too modest.

~~~~~ Calibration is the metric which is most interesting when
developing clinical prediction models. This is because in practice, risk
predictions from the model are given directly to a clinician who will
use the information to council patients and inform treatment decisions.
It is essential that these predictions are accurate (i.e., calibration
is good). Otherwise, the personal costs to the patient may be enormous.
Furthermore, it is entirely possible for a model to have great accuracy
and discrimination while calibration is poor \autocite{achilles}. Thus,
all three criteria must be considered when discussing the impact of
imbalance on the quality of clinical prediction models. This is rarely
the case and unfortunately, it is calibration that is often forgotten
\autocite{achilles}.

~~~~~Class imbalance is not unique to medical data sets. Thus,
literature focused on imbalance correction methods arises from many
disciplines. An abundance of imbalance corrections exist and are well
summarized by
\autocite{summary_b,summary_lp,summary_m,summary_h,summary_k}. Yet,
information regarding the effect of these corrections on model
calibration is sparse. One recent study demonstrated that implementing
imbalance corrections lead to dramatically deteriorated model
calibration, to the extent that no correction was recommended
\autocite{ruben}. In this study, models were developed using logistic
regression and penalized (ridge) logistic regression \autocite{ruben}.

~~~~~Motivated by the work of \textcite{ruben}, we must ensure that the
``cure'' is not worse than the disease. In our research, we aim to
assess the impact of imbalance corrections on model calibration for
prediction models trained with a wider variety of classification
algorithms including: linear classifiers, ensemble learning algorithms
and algorithms specifically designed to handle class imbalance.
Furthermore, we aim to answer the question: can imbalance corrections
improve overall model performance without comprising model calibration?

\begin{itemize}
\tightlist
\item
  expectations specified ?
\end{itemize}

\newpage

\hypertarget{analytic-strategy}{%
\subsubsection{2\textbar{} Analytic Strategy}\label{analytic-strategy}}

To provide a fair comparison of imbalance corrections a simulation study
will be designed and implemented.

\hypertarget{simulation-study}{%
\paragraph{2.1\textbar{} Simulation Study~}\label{simulation-study}}

\hfill\break
\hfill\break
We will adhere to the ADEMPT guidelines for the design and reporting of
our simulation study \autocite{tim_morris}.\\
The aim of the simulation study is to determine which pair(s) of
imbalance correction and classification algorithm can outperform the
classification algorithm alone.\\
\strut \\
Imbalanced data will be simulated to reflect 27 scenarios. The following
criteria will be varied: number of predictors, event fraction and sample
size. The number of predictors will vary through the set \{8,16,32\} and
event fraction, through the set \{0.5, 0.2, 0.02\}. The minimum sample
size for the prediction model (N) will be computed according to formulae
from \textcite{riley}. Sample size will then vary through the set
\{\(\frac{1}{2}\)N, N and 2N\}.\\
\strut \\
Under each scenario, 2000 data sets will be generated. More
specifically, test and training data will be generated such that the
training set is 10x larger than the test set.\\
Finally, each simulated data set will be analysed by 30 methods = 6
(classification algorithms) x 5 (imbalance corrections). Classification
algorithms will include: Logistic Regression, Support Vector Machine,
Random Forest, XG Boost, RUSBoost, and Easy Ensemble. Imbalance
corrections will include: random under sampling, random over sampling,
synthetic minority over sampling (SMOTE), SMOTE-ENN and no correction.\\
\strut \\
Finally, performance criteria will include: AUROC, MCC, accuracy,
sensitivity and specificity, calibration intercept and calibration
slope. Calibration intercept is the primary metric.\\

\begin{itemize}
\tightlist
\item
  something about reporting results ?
\end{itemize}

\hypertarget{software}{%
\paragraph{\texorpdfstring{2.2\textbar{} Software\\
}{2.2\textbar{} Software }}\label{software}}

\hfill\break
All analyses will be conducted using the open source statistical
software R \autocite{base}. Additionally, our simulation study is
expected to be quite computationally intensive. Therefore, we intend to
run the simulation using the high performance computers at the UMC.

\newpage

\hypertarget{references}{%
\subsubsection{References}\label{references}}

\printbibliography

\end{document}
