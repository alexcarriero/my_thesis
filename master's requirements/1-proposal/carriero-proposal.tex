% Options for packages loaded elsewhere
\PassOptionsToPackage{unicode}{hyperref}
\PassOptionsToPackage{hyphens}{url}
%
\documentclass[
  12pt,
]{article}
\usepackage{amsmath,amssymb}
\usepackage{lmodern}
\usepackage{iftex}
\ifPDFTeX
  \usepackage[T1]{fontenc}
  \usepackage[utf8]{inputenc}
  \usepackage{textcomp} % provide euro and other symbols
\else % if luatex or xetex
  \usepackage{unicode-math}
  \defaultfontfeatures{Scale=MatchLowercase}
  \defaultfontfeatures[\rmfamily]{Ligatures=TeX,Scale=1}
\fi
% Use upquote if available, for straight quotes in verbatim environments
\IfFileExists{upquote.sty}{\usepackage{upquote}}{}
\IfFileExists{microtype.sty}{% use microtype if available
  \usepackage[]{microtype}
  \UseMicrotypeSet[protrusion]{basicmath} % disable protrusion for tt fonts
}{}
\makeatletter
\@ifundefined{KOMAClassName}{% if non-KOMA class
  \IfFileExists{parskip.sty}{%
    \usepackage{parskip}
  }{% else
    \setlength{\parindent}{0pt}
    \setlength{\parskip}{6pt plus 2pt minus 1pt}}
}{% if KOMA class
  \KOMAoptions{parskip=half}}
\makeatother
\usepackage{xcolor}
\usepackage[margin=22mm]{geometry}
\usepackage{graphicx}
\makeatletter
\def\maxwidth{\ifdim\Gin@nat@width>\linewidth\linewidth\else\Gin@nat@width\fi}
\def\maxheight{\ifdim\Gin@nat@height>\textheight\textheight\else\Gin@nat@height\fi}
\makeatother
% Scale images if necessary, so that they will not overflow the page
% margins by default, and it is still possible to overwrite the defaults
% using explicit options in \includegraphics[width, height, ...]{}
\setkeys{Gin}{width=\maxwidth,height=\maxheight,keepaspectratio}
% Set default figure placement to htbp
\makeatletter
\def\fps@figure{htbp}
\makeatother
\setlength{\emergencystretch}{3em} % prevent overfull lines
\providecommand{\tightlist}{%
  \setlength{\itemsep}{0pt}\setlength{\parskip}{0pt}}
\setcounter{secnumdepth}{-\maxdimen} % remove section numbering
\usepackage[backend=biber,style=numeric,sorting=none]{biblatex}
\usepackage{booktabs}
\usepackage{longtable}
\usepackage{array}
\usepackage{multirow}
\usepackage{wrapfig}
\usepackage{float}
\usepackage{colortbl}
\usepackage{pdflscape}
\usepackage{tabu}
\usepackage{threeparttable}
\usepackage{threeparttablex}
\usepackage[normalem]{ulem}
\usepackage{makecell}
\usepackage{xcolor}
\ifLuaTeX
  \usepackage{selnolig}  % disable illegal ligatures
\fi
\usepackage[]{biblatex}
\addbibresource{carriero-proposal.bib}
\IfFileExists{bookmark.sty}{\usepackage{bookmark}}{\usepackage{hyperref}}
\IfFileExists{xurl.sty}{\usepackage{xurl}}{} % add URL line breaks if available
\urlstyle{same} % disable monospaced font for URLs
\hypersetup{
  pdftitle={Proposal: assessing the impact of class imbalance corrections on model calibration.},
  pdfauthor={Alex Carriero (9028757) },
  hidelinks,
  pdfcreator={LaTeX via pandoc}}

\title{\textbf{Proposal: assessing the impact of class imbalance
corrections on model calibration.}}
\author{\textbf{Alex Carriero (9028757) \vspace{4in}}}
\date{}

\begin{document}
\maketitle

\textbf{Date}: 13/10/2022\\
\strut \\
\textbf{Word Count}: 735\\
\strut \\
\textbf{Program}: Methodology and Statistics for Behavioral, Biomedical,
and Social Sciences.\\
\strut \\
\textbf{Supervisors}: Maarten van Smeden (Utrecht Medical Center
Utrecht), Ben van Calster (Leuven University and Leiden University
Medical Center) and Kim Luijken (Utrecht Medical Center Utrecht).

\textbf{Host Institution}: Julius Center for Health Science and Primary
Care, UMC.\\
\strut \\
\textbf{Candidate Journal}: Statistics in Medicine.\\
\strut \\
\textbf{FETC-approved}: 22-1809

\newpage

\hypertarget{introduction}{%
\subsubsection{1 \textbar{} Introduction}\label{introduction}}

~~~~~Prediction modelling in medicine is gaining increasing attention;
clinicians are often interested in predicting a patient's risk of
disease. Due to the (thankfully) rare nature of many diseases, the data
available to train clinical prediction models are often heavily
imbalanced (i.e., the number of patients in one class dramatically
outnumbers the other) \autocite{summary_m}. This is referred to as class
imbalance. Class imbalance is seen major problem in the field of machine
learning as it is known to degrade model performance \autocite{yu}.
Consequently, imbalance correction methodologies are proposed as a
solution \autocite{yu}.\\

~~~~~An ideal imbalance correction will improve all aspects of model
performance. These criteria include: classification accuracy,
discrimination and calibration. Accuracy refers to the proportion of
patients that a model classifies correctly (after a risk threshold is
imposed). Discrimination refers to a model's ability to yield higher
risk estimates for patients in the positive class than for those in the
negative class. Finally, calibration refers to the reliability of the
risk predictions themselves; for instance, a poorly calibrated model may
produce risk predictions that consistently over- or under-estimate
reality, or produce risk estimates which are too extreme (too close of 0
or 1) or too modest \autocite{achilles}.\\

~~~~~Class imbalance is not unique to medical data sets and literature
introducing imbalance correction methods arises from many disciplines.
An abundance of imbalance corrections exist and are summarized by
\autocite{summary_b,summary_lp,summary_m,summary_h,summary_k}.
Information regarding the effect of these corrections on model
calibration is sparse. In medicine, it is essential that model
calibration is assessed. This is because in practice, risk predictions
from the model are given directly to a clinician who will use the
information to council patients and inform treatment decisions. If a
model is poorly calibrated, the personal costs to the patient may be
enormous. It is entirely possible for a model to exhibit excellent
classification accuracy and discrimination while calibration is poor
\autocite{achilles}. Therefore, assessing only discrimination and
accuracy is insufficient.\\

~~~~~ Only one study has assessed the impact of imbalance corrections on
model calibration. \textcite{ruben} demonstrated that implementing
imbalance corrections lead to dramatically deteriorated model
calibration, to the extent that no correction was recommended
\autocite{ruben}. In this study, models were developed using logistic
regression and penalized (ridge) logistic regression \autocite{ruben}.
Motivated by the work of \textcite{ruben}, we must ensure that the
``cure'' is not worse than the disease. In our research, we aim to
assess the impact of imbalance corrections on model calibration for
prediction models trained with a wider variety of classification
algorithms including: linear classifiers, ensemble learning algorithms
and algorithms specifically designed to handle class imbalance.
Furthermore, we aim to answer the question: can imbalance corrections
improve overall model performance without comprising model calibration?

\hypertarget{analytic-strategy}{%
\subsubsection{2\textbar{} Analytic Strategy}\label{analytic-strategy}}

We will evaluate the performance of several imbalance corrections in a
simulation study. We will adhere to the ADEMP guidelines for the design
and reporting of our simulation \autocite{tim_morris}.

\hypertarget{simulation-study}{%
\paragraph{\texorpdfstring{2.1\textbar{} Simulation Study\\
}{2.1\textbar{} Simulation Study }}\label{simulation-study}}

\hfill\break
The aim of the simulation study is to determine which pair(s) of
imbalance correction and classification algorithm can outperform the
classification algorithms without imbalance corrections.\\
\strut \\
Imbalanced data will be simulated to reflect 27 scenarios. The following
criteria will be varied: number of predictors, event fraction and sample
size. The number of predictors will vary through the set \{8,16,32\} and
event fraction, through the set \{0.5, 0.2, 0.02\}. The minimum sample
size for the prediction model (N) will be computed according to formulae
from \textcite{riley}. Sample size will then vary through the set
\{\(\frac{1}{2}\)N, N and 2N\}.\\
\strut \\
Under each scenario, 2000 data sets will be generated. More
specifically, test and training data will be generated such that the
training set is 10x larger than the test set. Each simulated data set
will be analysed by 30 methods = 6 (classification algorithms) x 5
(imbalance corrections). The classification algorithms and imbalance
corrections we will include in our simulation are detailed in Table 1.

\begin{table}[!h]

\caption{\label{tab:unnamed-chunk-1}Classification algorithms and imbalance corrections to be evaluated.}
\centering
\begin{tabular}[t]{lll}
\toprule
Index & Classification Algorithms & Imbalance Corrections\\
\midrule
\cellcolor{gray!6}{1} & \cellcolor{gray!6}{Logistic Regression} & \cellcolor{gray!6}{None}\\
2 & Support Vector Machine & RUS (random under sampling)\\
\cellcolor{gray!6}{3} & \cellcolor{gray!6}{Random Forest} & \cellcolor{gray!6}{ROS (random over sampling)}\\
4 & XG Boost & SMOTE (synthetic majority over sampling)\\
\cellcolor{gray!6}{5} & \cellcolor{gray!6}{RUSBoost} & \cellcolor{gray!6}{SMOTE - ENN (SMOTE - edited nearest neighbours)}\\
\addlinespace
6 & Easy Ensemble & \\
\bottomrule
\end{tabular}
\end{table}

Finally, performance criteria will include measures of model
discrimination, accuracy and calibration. Discrimination will be
measured by area under the receiver operator curve (AUROC).
Classification accuracy by Matthew's correlation coefficient (MCC),
overall accuracy, sensitivity and specificity. Finally calibration will
be measured in terms of calibration intercept and calibration slope.

\hypertarget{software}{%
\paragraph{\texorpdfstring{2.2\textbar{} Software\\
}{2.2\textbar{} Software }}\label{software}}

\hfill\break
All analyses will be conducted using the open source statistical
software R \autocite{base}. Additionally, our simulation study is
expected to be quite computationally intensive. Therefore, we intend to
run the simulation using the high performance computers at the UMC.

\newpage

\printbibliography

\end{document}
